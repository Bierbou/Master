\chapter{Theory}\label{chap:th}
\noindent As the focus of this thesis lies on the experimental characterization of the whole setup, 
but especially on the source properties and the interferometric parts, 
the underlying theory of the physical principles of this work are presented restricted on the most essential basics.
The interested reader is referred for a further detailed overview about the topic to the recommendations at the beginning of each section,respectively.
\section{Interaction of X-rays with matter}\label{sec:ixm}
\subsection{Complex refraction-index}\label{subsec:cri}
\subsubsection{Attenuation}\label{subsec:att}
\subsubsection{Refraction}\label{subsec:ref}
\section{Grating-based phase contrast imaging}\label{sec:phase}
\subsection{Grating interferometer}\label{subsec:gi}
\subsection{Talbot effect}\label{te}
\subsection{Phase stepping routine}\label{subsec:stepp}







\section{Spatial system response} \label{sec:ssr}
\noindent In this section a short introduction to linear system theory is presented, 
focussing on the characterization of imaging systems. For a detailed insight the reader is referred to \ref{bib}. 
Here, only the basic quantities for real and radially symmetric response functions, representing the experimental, which is characterized, are treated.   
\subsection{Real- and frequency-space response functions } \label{subsec:ct}
\noindent Linear system theory is a common approach describing the spatial properties of imaging systems. 
The applicability of the superposition principle on an imaging system implies it's linearity, 
meaning the response to a linear combination of incitements, generally called input signals, 
is the same linear combination of the particular responses, called output signals. 
Assuming that the response of the linear system is also shift invariant, i.e. the system is linear and shift invariant (LSI), 
the output can be calculated by convolution\ref{bib}: 
\begin{equation}
i(x,y) = s(x,y) \otimes o(x,y) = \int_{-\infty}^{\infty}\int_{-\infty}^{\infty}s(x-x^{'},y-y^{'})o(x^{'},y^{'})dx^{'}dy^{'},
\end{equation}
$i(x,y)$ is the measured image, $s(x,y)$ is the point spread function (PSF), $o(x,y)$ describes the object, 
and $\otimes$ designates the two-dimensional convolution. Because of dealing with intensities here, 
all functions in real space are real functions, therefore the PSF is defined to be normalized to unity, i.e.:
\begin{equation}
\int_{-\infty}^{\infty}\int_{-\infty}^{\infty} s(x,y)dxdy = 1.
\end{equation}  
\subsection{Edge spread function} \label{subsec:esf}
The spatial system response on an edge-shaped input signal is generally defined as the systems edge spread function(ESF).

\begin{equation}
PSF_{System} = PSF_{Source} \otimes PSF_{Detector} = \mathcal{F^{-1}}[\mathcal{F}(PSF_{Source})\cdot \mathcal{F}(PSF_{Detector})]
\end{equation}

The magnification $M$ of the system is defined as the fraction of the source to detector distance over the source to sample distance, further called $SD$ and $SS$ e.g.
\begin{equation}
M = \frac{SD}{SS}
\end{equation}
Using geometry the true PSF of the source is spread over the detector screen by this magnification factor, so this factor has to be concerned
in the formulas. For convenience we just look at the one-dimensional case for the further steps. 
Normally a Gaussian shaped PSF's is assumed, whereby the sigma of the Gaussian has to be multiplied with the magnification factor minus one:

\begin{equation}
A_{System}e^{-\frac{1}{2}(\frac{x -b}{2\sigma_{system}})^{2}} = \mathcal{F^{-1}}[\mathcal{F}(A_{Source}e^{-\frac{1}{2}(\frac{x -b}{\sigma_{source}\cdot(M-1)})^{2}})
\cdot \mathcal{F}(A_{detector}e^{-\frac{1}{2}(\frac{x -b}{\sigma_{detector}})^{2}})]
\end{equation}
Solving this equation is straight forward, because the Fourier transform of a Gaussian is again a Gaussian, but with reciprocal width $\sigma$
The most important thing in this equation are the $sigma's$, respectively. 
For that reason we assume the amplitudes are one and the offset b is zero and we multiply both sides with the logarithmic function $\ln$.
Since $\ln{1} = 0$ and the logarithmic function is the inverted function of an exponential this yields to:
\begin{equation}
-\frac{1}{2}(\frac{x}{2\sigma_{system}})^{2} = -\frac{1}{2}(\frac{x}{2\sigma_{source}\cdot(M-1)})^{2}-\frac{1}{2}(\frac{x}{\sigma_{detector}})^{2}
\end{equation}

cancelling out the one half and dividing by x we get an equation where only the $sigma's$ stay:
\begin{equation}
(\frac{1}{\sigma_{system}})^{2} = (\frac{1}{\sigma_{source}\cdot(M-1)})^{2}-(\frac{1}{\sigma_{detector}})^{2}
\end{equation}
inverting the hole equation and taking the root we get an equation for the system's spread width:
\begin{equation}
 \sigma_{system} = \sqrt{{\sigma_{source}}^{2}\cdot(M-1)^{2}+{\sigma_{detector}}^{2}}
\end{equation}
 













