\chapter{The Setup}\label{chap:setup}
\begin{figure}[h]
	\begin{center}
		\includegraphics[width = 14.7 cm,keepaspectratio = true]{Setupcut}
	\end{center}
	\caption[Rendering of the whole setup]{\textit{Rendering of the setup which is characterized in this thesis. the optical table has a size of $2.4\ m \times 1\ m$. On the left side the micro-focus X-ray tube and on the other side of the table the Varian X-ray detector is mounted, which is usually used for clinical applications. In-between the important part of this setup the grating interferometer with stages between the gratings, handling the samples during the measurements, is placed. \tiny{(courtesy by Friedrich Prade)}}} 
	\label{setup}
\end{figure}
In this chapter a short introduction is given, about the individual parts and their properties of the setup. The setup was built up in administration of the biomedical physics group e17, by Friedrich Prade and Florian Schaff.  The whole setup consists more or less of three main parts, Source, Interferometer and Detector. The general structure is depicted in Fig:\ref{setup}. The micro-focus source is placed on the left and the X-ray detector on the right side. In-between the part,which needs the most space and where the \textquote{physics} happens, is mounted. Right in front of the Detector the analyser grating $G_{2}$ is attached on its own bench, in the middle of the table the phase grating $G_{1}$ is also placed on a bench, and right in front of the source the so called source grating $G_{0}$ is mounted. All gratings are preferably aligned perpendicular to the beam direction and with the same orientation of the grating-lines to each other, avoiding shadowing and that Moire-fringes show up. Between the gratings two different stages are placed handling the samples during the measurement, which can be controlled fully automatic, thus computer tomography measurements without interrupting the procedure are possible. In the next three sections the different parts are highlighted in a more detailed way, starting at the very beginning, the production of X-rays at the source, followed by the interacting part within the interferometer and at least the creation of the images at the detector. 
\section{The source} \label{sec:source}
The most indispensable part is of course the production of the required radiation itself. At this setup a so called \textquote{micro-focus} X-ray tube designed and crafted by \textquote{X-RAY WorX GmbH} is used, which provides some big improvements compared to commercial X-ray sources used for clinical applications, especially at the area of resolution restraints. The structure is in general like the structure of a conventional x-ray tube, but with additional parts e.g. an electron optic influencing the direction of the electrons generated by the cathode. For a complete overview of the properties and drawings of micro-focus tubes provided by this company see \citep{DatasheetX,Datashort,CAD,X-COMsoft}.   
\subsection{Dimensions and structure}\label{subsec:dimstruc}
The dimension and structure of the source differs just in the additional parts from conventional sources. It consists of a cathode filament heated by a high voltage power supply generating a electron beam, which is accelerated due to a electromagnetic field between cathode and anode, to the other end of the tube hitting a tungsten reflection target, which finally produces the X-ray radiation. In-between of these two main parts a additional part,a so called electron optic is placed. It consists of several magnetic coils which influences the trajectory of the electron beam. Four coils are responsible for the deflection of the whole beam in both x and y direction. The other coils are used focussing the electron beam, which is in general spread over a large area, to a very tiny area improving the resolution strength of the whole setup. A drawing of the dimension of the source and the electron optic in detail is shown in Fig. \ref{sourcerotfoc}. Using such a focussing equipment, it is inevitable to evacuate the whole tube avoiding flash-overs between filament and target. This is possible,because due to the strong compression of the particular electrons in the beam a \enquote{electron bridge} guiding current from cathode to anode can arise. Therefore in addition a vacuum turbo pump is mounted on top of the tube, providing a vacuum of the order of $1.5 \times 10^{-6} $ avoiding such flash-overs. As in conventional sources just about one percent of the electrons hitting the target, produces X-ray radiation. The remaining part heats the anode material. Thus especial for the case of a focussing source a cooling of the target is needed, avoiding that the target material melts down.      
\begin{figure}[h]
	\begin{center}
		\includegraphics[width = 14.7cm,keepaspectratio = true]{sourcerotfocus}
	\end{center}
	\caption[Dimensions of the source with focussing system]{\textit{Dimensions of the source and its components. The part highlighted in the red circle shows the focussing electron-objective, which improves the final resolution due to the fact scaling the spot size down to the micron range. The upper part of the electron optic consists of deflection magnets for the x-y plane. Subsequent to this are the focussing coils placed, which collimate the electron beam, denoted by the thin green line, to reach a really small impact point at the target. \tiny{(source: \copyright X-RAY WorX GmbH 2015 \url{http://www.x-ray-worx.com})}}}
	\label{sourcerotfoc}
\end{figure}
\subsection{properties}\label{subsec:sprop}
In difference to conventional X-ray tubes this tube is working over at broad tunable range from $20\ $kV up to $160\ $kV, but due to the design energy of the Talbot-Lau interferometer it operates in general at $60\ $kV. Depending on the adjusted acceleration voltage the power-output ranges from $1\ $W up to $300\ $W. The construction of the tube offers a almost infinite lifetime, because the filament, which is the only thing which usually breaks down, can be easily changed, thanks to its \enquote{open} architecture (see Fig.\ref{sourcerotfoc} right behind the opening level the filament holder is mounted). Another advantage compared to conventional sources with a spot size of $\approx 2\ \text{mm}^{2}$ is that the spot size of this source is reduced down to the micron range. Hence the spatial resolution is strongly improved, why it is possible resolving features down to sizes of $\approx 2\ \mu$m \citep{Datashort}. With the X-COM software \citep{X-COMsoft} also developed and traded by X-RAY WorX several adjustments are possible. One example for an additional feature is the possibility to \enquote{stepp} the beam in both x or y direction thanks to the linkage of the deflection magnets. With this feature a transition from a mechanical stepping procedure to a \enquote{magneto-stepping} for the phase contrast technique comes into range, which provides several advantages e.g. prevention of grating vibrations during the stepping process.
\begin{figure}[h]
	\begin{center}
		\includegraphics[width = 14.7cm,keepaspectratio = true]{setupdim}
	\end{center}
	\caption[Drawing of the relative distances between the particular parts of the setup]{\textit{Drawing of the relative distances in-between the setup and different sample positions. This drawing shows a symmetric setting of the gratings, thus the magnification factor is 2 and the analyser grating has twice the periodicity as the $\pi/2 $ phase grating. The gratings are also arranged to operate at the first fractional Talbot distance. The blue extended arrows indicate the possible positions for the sample between the gratings.}}
	\label{setupdim}
\end{figure} 
\section[Sample handling]{Sample handling between the gratings}\label{sec:samplehandling}
As depicted in Fig.\ref{setupdim}, the Talbot-Lau interferometer itself captures the main part of the setup length, starting about eight cm behind the source, with the source grating $G_{0}$, which is an absorption type with a grating period of $10\, \mu$m and a height of $150??\, \mu$m. At the other side just two cm in front of the flat-panel detector the analyser grating $G_{2}$ with the same periodicity of $10 \ \mu m$ as $G_{0}$ and a grating height of $160-170\, \mu$m, thus also and absorption grating, is placed. Due to the high absorbency and the easy handling these two gratings are made of gold as described in \ref{subsec: gt}. Accounting the magnification factor of two for a symmetric setup, the phase grating $G_{1}$ with a periodicity of $5\, \mu$m and a nickel height of $8\, \mu$m has to be put right in the middle between the other gratings so the distance between the gratings is generally $92.5\, $cm, respectively. The gratings are not set arbitrarily to these distances, but they are put there on purpose, accounting the first fractional Talbot distance for a design energy of $45\,$keV. In-between the gratings two different kinds of sample holders are placed with different properties. The first one between $G_{0}$ and $G_{1}$ is an Euler cradle. the main advantage of this stage is, that the sample can be rotated around its own axis and furthermore rotated around the beam axis as well. The second stage is a usual tomography stage which is put between $G_{1}$ and $G_{2}$. Depending on the expected results and the sample is put on one of these two stages. So as example if a bigger magnification is needed resolving small detail within a sample the Euler cradle is recommended. On the other hand if one is interested in the induced phase shift one has ot assure to get close to the phase grating, because the sensitivity enhances getting closer to it. But for this case both stages are feasible, because it makes no difference if the sample is placed in front or behind the phase grating. 
\section{detector properties}
The detector used for imaging is a PaxScan 2520 DX digital flat-panel detector generally used for dental cases. The detector is developed and traded by Varian medical systems and is based on amorphous silicon technology with \gls{csi} as conversion material on top. The main advantage of this material are radiation hardness $> 1$Mrad, a broad input energy range from about $40 - 160\,$kVp, good low dose performance, immunity for single photon events within the substrate and proven 3-D soft tissue capability. The total pixel area is $19.5\times 24.4\,$cm with a pixel size of $127\, \mu\text{m}^{2}$ this yields to $1536 \times 1920$ pixel. The limiting resolution is stated to be $3.94\,$lp/mm or in other words $253.81\, \mu$m in both x- and y-direction \citep{Paxscan}. A picture of the detector is shown in Fig.\ref{paxscan}. For reasons of protection a $2.5\,$mm thick carbon fibre plate combined with aluminium is placed in front of the active detector area. Right behind this plate the active crystal layer with $\approx 9$mm thickness is placed.    
%\begin{figure}[h]
%	\begin{center}
%		\includegraphics[width = 14.7 cm,keepaspectratio = true]{paxscan}
%	\end{center}
%	\caption[short]{title}
%	\label{paxscan}
%\end{figure}

















