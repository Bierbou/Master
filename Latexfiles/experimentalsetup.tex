\chapter{The Setup}\label{chap:setup}

\begin{figure}[h]
	\begin{center}
		\includegraphics[width = 12.75 cm,keepaspectratio =true]{Setupcut}
	\end{center}
	\caption[Schematic rendering of the whole setup]{\textit{Rendering of the setup characterized in this thesis. The optical table has a size of $2.4\ m \times 1\ m$. A micro-focus tube, seen on the left, is used as source. A Varian Paxscan 2520DX X-ray detector is mounted on the other side of the table. These kind of detectors are commonly used for clinical applications. Three gratings, $G_{0}$, $G_{1}$, $G_{2}$, that form a Talbot-Lau interferometer are positioned between source and detector. Samples can be mounted on two stages, either between $G_{0}$ and $G_{1}$, or between $G_{1}$ and $G_{2}$. Courtesy by Friedrich Prade.}} 
	\label{setup}
\end{figure}

\newpage
In this chapter a short introduction about the individual parts of the setup and their properties is given. The setup was built up by Friedrich Prade and Florian Schaff of the biomedical physics group E17 in the TUM Physics department. The whole setup consists of three main parts: source, interferometer and detector. A general overview is depicted in Figure \ref{setup}. A micro-focus source is placed on the left and the X-ray detector on the right side of an optical table. A Talbot-Lau interferometer consisting of three gratings is installed in between the source and the detector. The analyser grating $G_{2}$ is placed right in front of the Detector, the phase grating $G_{1}$ can be seen in the middle of the table, and right in front of the source the so called source grating $G_{0}$ is mounted. All gratings are preferably aligned perpendicular to the beam direction and with the same orientation of the grating-lines to each other. Between the gratings two different sample stages are placed, which allow for fully motorized sample positioning. In the next three sections the different parts are highlighted in a more detailed way, starting with the production of X-rays at the source, followed by the interferometer and lastly the creation of the images at the detector. 
\section{The source} \label{sec:source}
The most indispensable part is of course the production of the required radiation itself. At this setup a so called \textquote{micro-focus} X-ray tube designed and crafted by \textquote{X-RAY WorX GmbH} is used. A micro-focus X-ray tube provides some big improvements compared to commercial X-ray sources used for clinical applications, especially with regards to achievable resolutions. The structure is in general to that of a conventional x-ray tube, but with additional parts, including an electron optic to focus the electrons generated by the cathode. For a complete overview of the properties and drawings of micro-focus tubes provided by this company see \citep{DatasheetX,Datashort,CAD,X-COMsoft}.   
\subsection{Dimensions and structure}\label{subsec:dimstruc}
The dimension and structure of the source differs just in the additional parts from conventional sources. It consists of a heated cathode filament generating an electron beam. These electrons are then accelerated by and electric potential applied between cathode and anode. At the other end of the tube the electrons hit a tungsten reflection target, which produces the X-ray radiation. In-between these two main parts an additional part, a so called electron optic is placed. It consists of several coils which influences the trajectory of the electron beam. Four coils are responsible for the deflection of the whole beam in vertical and horizontal direction. The other coils are used for focussing the electron beam. It is in general spread over a large area, before being focussed to a very tiny area. This improves the resolution of the whole setup. A drawing of the source and the electron optic in detail is shown in Figure \ref{sourcerotfoc}. Due to absorption of electrons by air the tube has to be evacuated. For that reason a vacuum pump is mounted in addition on top of the source, providing a vacuum in the order of $1.5 \times 10^{-6} $, because in contrast to usual tubes, this micro-focus X-ray tube is a so called \textquote{open tube}.  
%Using such a focussing equipment, it is inevitable to evacuate the whole tube avoiding flash-overs between filament and target. This is possible,because due to the strong compression of the particular electrons in the beam a \enquote{electron bridge} guiding current from cathode to anode can arise. Therefore in addition a vacuum turbo pump is mounted on top of the tube,  avoiding such flash-overs. 
As in conventional sources just about 1\% of the delivered power, produces X-ray radiation \citep{Lehnertz}. The remaining part heats the anode material. Thus especially for the case of a focussing source a cooling of the target is needed, to avoid the target material melting down.      
\begin{figure}[h]
	\begin{center}
		\includegraphics[width = 14.7cm,keepaspectratio = true]{sourcerotfocus}
	\end{center}
	\caption[Dimensions of the source with focussing system]{\textit{The source and its components. The length in the scheme are given in millimetre. The part highlighted in the red circle shows the focussing electron-objective, which improves the final resolution by scaling the spot size down to the micron range. The upper part of the electron optic consists of deflection coils for the vertical and horizontal direction. Subsequent to this the focussing coils are placed, which collimate the electron beam, denoted by the thin green line, to reach a small impact point at the target. \\source: \copyright X-RAY WorX GmbH 2015 \url{http://www.x-ray-worx.com}}}
	\label{sourcerotfoc}
\end{figure}
\subsection{Properties of the source}\label{subsec:sprop}
This tube is able to work over a broad tunable range from $20\,$kV up to $160\,$kV, but due to the design energy of the Talbot-Lau interferometer it is in general operated at $60\,$kV. Depending on the acceleration voltage and electron current the power-output ranges from $1\,$W up to $300\,$W. The construction of the tube offers a long lifetime, because the wearing parts, filament and target, can be changed easily changed, thanks to its \enquote{open} architecture (see Figure \ref{sourcerotfoc} right behind the opening level the filament holder is mounted). With the X-COM software \citep{X-COMsoft} also developed and distributed by X-RAY WorX several adjustments are possible. One example for an additional feature is the possibility to manually adjust the beam in vertical and horizontal direction thanks to the linkage of the deflection magnets. This feature combined with the small focus spot allows for the transition from a mechanical stepping procedure to an electromagnetic one, in the context of Talbot interferometry. One possible advantage of this could be e.g. prevention of grating vibrations during the stepping process \citep{Harmon2015}.
\begin{figure}[h]
	\begin{center}
		\includegraphics[width = 14.7cm,keepaspectratio = true]{setupdim}
	\end{center}
	\caption[Drawing of the relative distances between the particular parts of the setup]{\textit{Drawing of the relative distances in-between the setup and different sample positions. This drawing shows a symmetric setting of the gratings, thus the magnification factor is 2 and the analyser grating has twice the periodicity of the $\pi/2 $ phase grating. The gratings are also arranged to operate at the first fractional Talbot distance. The blue extended arrows indicate the possible positions for the sample between the gratings.}}
	\label{setupdim}
\end{figure} 
\section[Sample handling]{Sample handling between the gratings}\label{sec:samplehandling}
As depicted in Figure \ref{setupdim}, the Talbot-Lau interferometer itself occupies the main part of the setup length, starting about $8\,$cm behind the source, with the source grating $G_{0}$. It is a gold absorption grating with a period of $10\,\mu$m and a height of $150\,\mu$m. At the other side just $2\,$cm in front of the flat-panel detector the analyser grating $G_{2}$ with the same periodicity of $10\,\mu m$ as $G_{0}$ and a grating height of $160-170\,\mu$m gold, thus also and absorption grating, is placed. Due to the high absorbency and the easy handling these two gratings are made of gold as described in \ref{subsec: gt}. Accounting the magnification factor of two for a symmetric setup, the phase grating $G_{1}$ with a periodicity of $5\,\mu$m and a nickel height of $8\,\mu$m has to be put right in the middle between the other gratings. This leads to a distance to each grating of $92.5\, $cm. The gratings are not set arbitrarily to these distances, but they are put there on purpose, accounting for the first fractional Talbot distance at a design energy of $45\,$keV. In between the gratings two different kinds of sample holders are placed with different properties. The first one between $G_{0}$ and $G_{1}$ is an Eulerian cradle. The main advantage of this stage is that the sample can be rotated around its own axis and furthermore rotated around the beam axis as well. The second stage is a usual tomography stage which is put between $G_{1}$ and $G_{2}$. Depending on the sample, it is put on one of these two stages. For example, if a bigger magnification is needed resolving small details within a sample the Euler cradle is recommended. On the other hand, if one is interested in the induced phase shift one has to assure to get close to the phase grating, because the sensitivity increases closer to $G_{1}$. For this case both stages are feasible, because it makes no difference if the sample is placed in front or behind the phase grating \citep{Yashiro2008}. 
\section{Detector properties}
The detector used for imaging is a PaxScan 2520 DX digital flat-panel detector generally used for dental and industrial cases. The detector is developed and sold by Varian medical systems and is based on amorphous silicon technology with \gls{csi} as conversion material on top \citep{Paxscan}. The main advantages of this material are radiation hardness $> 1$Mrad, a broad input energy range from about $40 - 160\,$kVp, good low dose performance, immunity for single photon events within the substrate and proven 3-D soft tissue capability. The total pixel area is $19.5\times 24.4\,$cm. With a pixel size of $127\, \mu\text{m}^{2}$ this amounts to $1536 \times 1920$ pixels. The limiting resolution is stated to be $3.94\,$lp/mm or in other words $253.81\, \mu$m in vertical and horizontal direction \citep{Paxscan}. A picture of the detector is shown in Figure \ref{paxscan}. For reasons of protection a $2.5\,$mm thick carbon fibre plate combined with aluminium is placed in front of the active detector area. Right behind this plate the active crystal layer with $\approx 9\,$mm thickness is placed.    
\begin{figure}[h]
	\begin{center}
		\includegraphics[width = 14.7 cm,keepaspectratio = true]{paxscan}
	\end{center}
	\caption[Varian detector properties]{\textit{Image and properties of the PaxScan detector 2520 DX. Left hand: image of the PaxScan detector screen with the black carbon-fibre shielding in front of the active detector area. Right hand: Structure and dimensions of the detector from different perspectives. source: \citep{Paxscan} }}
	\label{paxscan}
\end{figure}

















