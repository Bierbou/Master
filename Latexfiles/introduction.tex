\chapter{Introduction}\label{chap:intro}
The ability of X-rays to penetrate matter makes them an indispensable tool in both medical diagnostics as well as non destructive testing.
%The huge advantages using X-rays are nowadays indispensable, for issues as non invasive medical investigations or non destructive testing of various materials in the industry. 
Especially for industrial applications, where radiation dose and measurement time are not nearly as big of a defining factor as with all medical applications, a high spatial resolution is often times desired. This can be achieved using X-ray tubes with very small spot-sizes.
%The development of conventional X-ray tubes, with smaller and smaller spot-sizes opens new possibilities to lower the resolution limit of setups using such conventional tubes and thus shelter great potential for various measurement methods. 
In addition the recent invention of interferometer setups for X-ray phase-contrast imaging \citep{David2002} can be used to enhance image quality, both for industrial and clinical applications. Subsequent the combination of both, small spot-sizes and such interferometer setups is capable to further increase the image quality. \\
% tremendously enhanced the interest of imaging with X-rays.\\ 

In order to be able to optimize imaging results, the exact investigation of the properties of the measurement setup is just as important as the investigation of samples itself. An imaging system consists of several different parts that need to be characterized, e.g the source, detector and possible additional parts such as a grating interferometer. This may help understand the origin of artefacts, which are often induced by intrinsic properties of the setup. Subsequently, the knowledge of e.g. the behaviour of the \gls{psf} of the entire system can be used to correct the images for some artefacts.\\ 

Hence, this thesis is based on the characterization and optimization of a high-resolution X-ray interferometer setup for material research. The setup consists of a micro-focus X-ray tube \citep{DatasheetX}, a Varian flat-panel detector \citep{Paxscan}, as used for medical imaging, and a Talbot-Lau grating-interferometer \citep{Pfeiffer2006}. A grating-interferometer can be built by a minimum of two gratings, whereupon in the case of a Talbot-Lau interferometer a third grating is added in front of the source to reach sufficiently high spatial coherence. Without this third grating the Talbot pattern vanishes caused by the extent of the source size.\\ 

The thesis is mainly focussed on the characterization of the source properties, especially on the determination of the spot-size with different techniques, the influence of the interferometer parts onto the flux and the source spectrum, and the investigation of new measurement methods which can be derived from the results obtained by the previous characterization. The main advantage of such micro-focus tubes is the small spot-size in the $\mu$m range, which firstly allows for measurements with great geometrical magnification, without losing resolution due to source induced blurring. Secondly, a small source-spot should in principle provide sufficient spatial coherence, so that the source grating can be omitted without losing the ability to perform interferometry measurements. Hence, the determination of the exact spot-size for various X-ray tube settings is of great interest.
%to be able to omit the grating in front of the source, without loosing the ability to detect the refraction of the X-rays. Hence the determination of the sources spot-size is for instance very important, because it is the main limiting factor, which has to be investigated in the first place to give a statement about the applicability omitting the so called source-grating.

\section{Outline}
At the beginning of this thesis, a short introduction about the theory for the different measurements and applied techniques is given. Subsequently, a detailed description of the setup characterized in this thesis is given. Afterwards, the stability of the source with respect to power-output was characterized. Following, the behaviour of the source size for various different acceleration voltages and powers was determined. The spectrum of the source and influences of the interferometer gratings onto the spectrum were investigated followed by the measurement of the energy resolved visibility of the entire setup. Finally, the obtained results were used to proof investigate the possibility of operating the setup without a source grating.
% the ability of two applications omitting the source grating.    
              














