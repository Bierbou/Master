\chapter{Summary and Outlook}\label{chap:sum}
Here, the most important results and their implications for measurements at the characterized setup are are summarized. %At the end of the chapter an out-look for further investigations and measurements is given. 
The measurements presented in the different chapters yielded the following main results:\\

The time and power stability measurements showed a stable production of X-rays with an almost constant intensity over a long time scale. In addition, the increase of the intensity with varied power at a fixed electron energy showed a linear behaviour, except between a power of $25\,$W and $45\,$W, see Figure \ref{powerdep}. This can possibly be explained by the change of the focussing of the electron beam at this power, in order to prevent damage to the tungsten target.\\ 
	
The characterization of the X-ray tubes focal spot resulted in an ellipse shaped spot size. The dependence of the spot size on the tube power can be separated into three regions: a low power region with a nearly constant small spot size up to $30\,$W, a region of rapid increase of the spot size up to $50\,$W, and again a region of nearly constant spot size for powers higher than $50\,$W. The vertical spot size is always much smaller than the spot-size in horizontal direction, which is mainly caused by the geometry of the source c.f. Figure \ref{sourceellipse}. The \gls{psf} of the detector is assumed to be constant, thus the \gls{ssr} of the whole system is only dependent on the behaviour of the source.\\ 
%In the region below $30\,$W the source size is almost constant in horizontal as well in vertical direction and especially for the vertical direction very narrow, below $4\, \mu$m !\\
%In the transition region between $30$ and $50\,$W, the spot-size rapidly increases due to the need of readjustment of the sources electron-optic.\\
%in the region above $50\,$W, the spot-size increases slowly with an almost linear behaviour.\\
	
The main results of the measurement of the source spectrum and the influences of different setup parts can be concluded in two main statements. The X-ray spectrum of the source is strongly influenced especially by the two absorption gratings, whereupon the difference of the two resulting spectra can not be completely explained, yet cf. Figure \ref{specgrat}. Secondly, the resulting energy-visibility-map of the second part of chapter \ref{chap:spectra} gives a very good overview over the visibilities dependency on the energy. The maximum of the obtained curve perfectly hits the design energy of the interferometer, which improves the results.\\
	
Finally, the small source size in the low power region allows for operating the grating interferometer without a source grating as presented in chapter\ref{chap:applic}. It was shown, that the setup can be operated in two-grating configuration. The resulting visibility was sufficient to perform imaging experiments efficiently, and the sensitivity was comparable to that of the standard setup.
%The first example showed the possibility to omit the source grating, but keep the detectability of the different signals at almost the same level as for setups with a source grating. \\
Furthermore, this configuration opens up the possibility to perform electromagnetic stepping, rather than mechanical stepping of the gratings. Electromagnetic stepping of the source spot hast the advantage that no parts need to be moved mechanically, which often times leads to vibrations. Additionally, the drawback of the source grating absorbing half the intensity can be avoided.\\ 
%The second example showed an auspicious approach to make the transition from a mechanical stepping towards a fully electromagnetic stepping, yielding to the possibility to measure faster due to the doubled intensity at the same power and last but not least to reduce grating-vibrations induced by the mechanical stepping of the grating.\\      

%For future aspects, there are several additional measurements which can be used to understand the properties of the complete setup and their respective origins in more detail, as for example the investigation of the origin of the drift of the gratings or the thermal influences on different parts of the setup and thus onto the image quality. In addition  the electromagnetic stepping showed up good results. Hence, it is worth to further investigate possibilities by additional measurements to get a better understanding of the whole procedure. A subsequent step can also be testing the approach with different gratings to further improve the ability of the whole technique.

The use of a microfocus tube has the advantage that high resolutions can be achieved. This is of great interest for material science, where time limitations are of secondary concern. For a grating interferometer, a small spot size allows to omit the source grating, which has several advantages, e.g. higher useable flux. Currently, the achieveable visibility of a two-grating setup is still below that of a comparable three-grating setup. However, the possibility to perform electromagnetic stepping showed great promise. Further research could be done to optimized the setup for this type of measurement procedure.
 











