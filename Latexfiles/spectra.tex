\chapter{Spectra measurements}\label{chap:spectra}
In this chapter the spectrum of the X-ray source itself and the spectrum which remains after passing through the different gratings of the interferometer is characterized. In addition, the phase-stepping is measured energy resolved, which allows to quantify at which energy the visibility of the stepping is maximized.
\section{Measurement procedure of the different spectra}\label{sec:sourcespectra}
In this section, the substantial procedure and the different parameter of the measurements are presented. First of all, measuring any spectrum an energy-resolved detector is needed. Here, the detector used for the measurements is an AMPTEC XR-100T-CdTe X-ray detector provided by 'GE global research'. The active-layer of this detector consists of \gls{cdte}, which is excellently sensitive for the detection of X-ray photons with energies up to $100\,$keV. The active area of the detector is $5\times5\, \text{mm}^{2}$, has a thickness of $1\,$mm and is covered by a Beryllium window due to reasons of protection. To prevent the saturation of the detector, the active area is additionally covered by a lead pin-hole collimator with a diameter of $1\,$mm. As the focus of this chapter lies on the characterization of the spectra, the properties of the detector are not presented in further detail, hence the interested reader is referred to \cite{amptecdata,Amptec}. \\

As mentioned in section \ref{sec:source}, the target of the source consists of tungsten, which is a metal with high density. To be able to calibrate the detector before the measurements, it is indispensable to know at least the energies of two characteristic emission peaks of the target material. The detector only consists of channels at which the different charges, produced by charge separation of the particular photons with distinct energy impacting on the active-layer are stored. Each of these channels stores a distinct amount of charge, which afterwards can be converted  with aid of the characteristic peaks to eV units. The $k_{\alpha2}$ and $k_{\beta1}$ emission peaks of Tungsten used for the calibration appear at energies of $57.981$ and $67.244\,$keV, respectively. The values were looked up at \cite{Thompson}. Additionally, in contrast to conventional X-ray sources at which the exit window of the tube consists of Beryllium to keep the vacuum and to provide nearly non-dissipative emission of the X-rays, the window of this source is made of Aluminium. Due to that, low energy photons are filtered out of the emitted X-ray beam, which changes the shape of the spectrum. This fact has to be kept in mind for the later discussion of the obtained spectra. For the measurements the detector is placed at the same position as the flat-panel detector. After the calibration, the 'raw' spectrum of the source is measured at first at different peak energies of $60$ and $80\,$kVp with a power of $10\,$W and a measurement time of $300$ seconds, to get a clue of the behaviour of the spectrum within theses two different energy ranges. Subsequent, the different interferometer gratings are put step by step between source and detector with the same parameter settings, to characterize the influence of the different gratings onto the spectrum, which is usually recorded by the flat-panel detector. To come as close as possible to the spectrum which is usually recorded, the gratings are aligned to each other in the same way as before a usual measurement. In addition to these measurements, the spectrum of each particular step of the phase-stepping process is measured. These spectra are processed afterwards in the same manner as described in section \ref{sec:phase}. This offers the possibility to quantify the energies of the spectrum which provide the best fringe-contrast and hence the best visibility for the set of these interferometer gratings. The measurement is performed at a peak-energy of $100\,$kVp a power of $100\,$W and an integration time of $60$ seconds at each step.\\ 

To obtain the results in a more quantitative way an additional treatment is needed. The spectrum of the source is exposed on its way to the detector to several external influences, as interaction with air absorption at the Beryllium window of the detector etc. Therefore, the spectra have to be corrected with the linear attenuation coefficients $\mu$  and the thickness of the respective materials to obtain the 'true' spectrum. Secondary, the spectra are corrected with the linear absorption coefficients of \gls{cdte} and \gls{csi}, which is the active-layer material of the flat-panel detector, to show the spectrum 'seen' by the standard Paxscan flat-panel detector of the setup.    
\section{Results and discussion}\label{sec:specresults}
In this section the outcome of the measurements described above is explained and discussed. First of all, it is important to know that \gls{cdteg} has some special properties which are hardly treatable. Besides the almost perfect efficiency over a broad energy range, the material has a negative side-effect, which has to kept in mind for the interpretation of the measured spectra. This side effect are so called \textit{escape events}. These events appear, when X-ray photons with energies above the K edges of Cd and Te, $26.704$ and $31.8\,$keV, respectively, undergo photoelectric interaction within the active-layer material. After the interaction the Cd and Te atoms are left at an excited state. While relaxing back into the ground state the atoms emit characteristic X-rays without any predominant direction. Hence it happens that some of theses photons are emitted in reverse direction and thus are able to leave the detector material. If this is the case, an X-ray photon that deposits lets say an energy of $60\,$keV is only counted as a photon with an energy between $33-36\,$keV or $28-33\,$keV depending on the atom of interaction. Due to that there are much more events counted around the K edges of the detector material and additionally photons with energies just above the absorption edges are shifted to low energies, which distorts the whole spectrum. For a more detailed insight in the different effects see \citep{Redus2008}.\\
To overcome this there are some software tools provided by \textit{AMPTEC Inc.}, but they are to expensive to buy them for a few corrections.% \clearpage
\begin{figure}
	\begin{center}
		\includegraphics[width= 14.7 cm,keepaspectratio = true]{sourcespectra6080}
	\end{center}
	\caption[Corrected and uncorrected source spectra for different peak-energies]{\textit{Comparison of the corrected and uncorrected source spectra for different peak energies. 'Corrected' in this case  means conversion of the measured spectrum into the spectrum which is recorded by the flat-panel detector. a) Comparison of the measured and the corrected spectrum at an energy of $60\,$kVp, with no characteristic tungsten peaks. b) Measured and corrected source spectrum at an energy of $80\,$kVp with visible characteristic tungsten peaks at $ 58\ \text{and}\ 67\,$kVp. The increase of the intensity with its maximum around $23\,$keV is induced by intrinsic properties of the \gls{cdte} detector material.}}
	\label{spectra6080}
\end{figure}    
\subsection{Source spectra at different photon- energies}\label{subsec:spectra}
To have an impression of how the source spectrum looks like for different acceleration voltages, spectra at $60\,$kVp and $80\,$kVp are compared. The results are shown in Figure \ref{spectra6080}. Here, the spectra measured by the \gls{cdte} detector are denoted in both plots in green and the corrected spectra with respect to the \gls{csi} of the flat-panel detector are denoted by the blue line. The absorption of X-rays due to interaction with air molecules was not accounted for, because the absorption by air is given in every measurement and thus treated as an 'intrinsic' property of the setup. In Figure \ref{spectra6080}, a) shows the spectrum at $60\,$kVp and b) at $80\,$kVp. The difference between the corrected and uncorrected spectrum is almost not visible in a), whereas in b) the difference gets bigger, as the efficiency of \gls{csi} compared to \gls{cdte} is similar, but its getting worse at higher photon energies. The only difference clearly visible in both plots is the intensity drop at $33\,$keV, which is caused by an absorption edge of \gls{csi} at this energy. Due to this, the values of $\mu_{CsI}$ drop down close to zero, which causes a big change in the correction factor based on the 'Lambert-Beer Law'. The Peaks with the highest intensity in both plots are caused by the escape events explained at the beginning of the section and can unfortunately not be removed with easy methods. In addition, there should be almost no intensity measured below $\approx 15\,$keV, because the Aluminium window of the source functions as a filter for low energy photons . In comparison the spectra for the different acceleration voltages look quite similar besides the fact, that in a) none of the characteristic tungsten peaks are visible in contrast to b) where the peaks appear.      
%- different max photon energy -> evolution of spectrum for diff max energies\\
\begin{figure}%[h]
	\begin{center}
		\includegraphics[width= 14.7 cm,keepaspectratio = true]{gratingspectra}
	\end{center}
	\caption[Influence of the interferometer gratings onto the source spectrum]{\textit{Illustration of the influence of different interferometer-gratings and their combinations onto the source spectrum at different maximum photon energies. a) Comparison of the reduction of intensity onto the source spectrum at $60\,$kVp, caused by the attenuation due to the different interferometer gratings. b) Comparison of the same spectra at an energy of $80\,$kVp. A striking feature is the strong intensity reduction at the spectra with the phase grating inside the beam. Possible explanation are 'escape events' an intrinsic property of the detector material.}}
	\label{specgrat}
\end{figure}
\subsection{Influence of the gratings onto the source-spectrum}\label{subsec:gratspec}
After comparison of the 'raw' source spectrum with the \gls{csi} weighted spectrum, the influence of the different interferometer gratings and combinations of them is investigated. The changes of the spectra at different peak energies are illustrated in Figure \ref{specgrat}. For a better comparison the respective corrected 'raw' spectrum of Figure \ref{spectra6080} is added to the particular plots. a) shows the influence of the gratings at $60\,$kVp and b) at an energy of $80\,$kVp, respectively. The measurement time and power were kept at the same values as mentioned above. Both plots show a drastic reduction of intensity regardless of the respective grating put into the beam. One very interesting result strikes the eye comparing the spectra for $G_{0}$ and $G_{1}$ in both plots. Usually one expects a much better reduction at higher energies for the case of the source grating, as the $G_{0}$ is an gold absorption grating and the $G_{1}$ is a nickel phase grating with a height of only $8\, \mu$m. A possible explanation is maybe again the appearance of escape events. As the occurrence of these events has a highly statistical origin it can happen, that during one measurement more photons 'escaped' than in another measurement. This could explain the decrease of the spectrum with the $G_{1}$ at higher energies, because there is an increase of intensity around the energies of the escape peaks of $27$ and $31\,$keV. In contrast, the source grating absorbs much more up to higher energies, that's why the region around the escape peaks is flattened, because much more photons with such energies are filtered out by the gold-grating. In addition it is also possible, that the distance from source to grating plays an important role, because the content of gold inside the beam is much less for $G_{0}$ than for the analyser grating $G_{2}$. This explains the shift of the spectrum  with $G_{2}$ over the whole energy range to lower intensities and the higher intensities at higher energies of the spectrum with $G_{1}$ compared to the spectrum with $G_{1}$. The absorption strength of the nickel phase grating $G_{1}$ becomes very clear by comparison of the spectrum with solely $G_{1}$ and the spectrum with the combination of $G_{1}$ and $G_{2}$. The result is, that the phase grating has only a small influence and also just at lower photon energies. This improves the assumption for phase measurement, that the phase grating is only affecting the phase of the intensity-pattern but not intensity itself, whereupon in contrast the loss of intensity compared to the 'raw' spectrum is proportionally strong. It is not surprising that the spectrum for the all three gratings inside the beam is strongly reduced, but this has also a positive effect, because the maximum of intensity is shifted to the design energy of the interferometer of about $45\,$keV. 
\subsection{Energy resolved phase-stepping and Energy-visibility map}\label{subsec:ephasestep}
In this part the results of the spectra measured at each particular step of the phase-stepping routine are presented. The big advantage of the measurement with a energy resolved detector for this technique, is the possibility to resolve the intensity variation during each phase step for each particular energy, whereupon in contrast with a conventional integrating detector only the averaged variation can be observed. 
\begin{figure}%[h]
	\begin{center}
		\includegraphics[width= 10.89 cm,keepaspectratio = true]{steppingenergyvisimap}
	\end{center}
	\caption[Energy resolved stepping and energy-visibility map]{\textit{Illustration of the results of the phase-stepping procedure and related visibility. a) Energy resolved plot of the intensities during the phase-stepping. b) Resulting energy-visibility map obtained from the phase-stepping in a). The highest visibility occurs around $35-45\,$keV, which is the design energy of the interferometer. The increase of visibility above $80\,$keV is due to the gold absorption-edge, which enhances the photon-absorption of the analyser grating up to higher photon-energies.}}
	\label{stepvisi}
\end{figure}
The spectra at the different grating positions are depicted in Figure \ref{stepvisi} a). In this graph, due to reasons of simplicity, the uncorrected spectra are presented, as is would make no difference with respect to the variations among each other, because the correction affects each spectrum in the same manner and thus would only rescales the whole graph. The graph shows the $8$ steps over one $G_{1}$-grating period of $5\, \mu$m, which are usually used to retrieve the induced phase shift of an object onto the X-ray beam. Here, the focus lies on the difference between the particular steps, whereupon at the first glance only one energy region indicates big variations. This region lies between $30$ and $55\,$keV which is no wonder, because the interferometers design energy for the first fractional Talbot distance is $45\,$keV, which thus is explained perfectly. To get a more quantitative insight at which energies big variations occur, the bunch of steps is processed with the same algorithm explained in section \ref{sec:phase}. As result the visibility at each energy can be obtained. The result is shown in Figure \ref{stepvisi} b). The first thing which strikes the eye is the fact, that there are three regions with feasible visibility, which also means that in a) variations also at other energies occur. One of the two other region appears at low energies which is not that easy to explain, because at this region there should be almost no intensity due to the filtering of the sources Aluminium window. A possible explanation is the influence of 'low energy events' induced by the detectors material properties. Nevertheless, this region coincides with the region of the third fractional Talbot distance of the interferometer, so there should be some visibility. The third region with practical visibility arises after an energy of $80\,$keV. The increase after this energy has a very simple explanation. The visibility enhances after this energy, due to the absorption edge of gold, the absorption of gold is strongly increased, which also increases the variation between the particular spectra. The reason why the two additional regions are not visible in a) is quite simple. The visibility is determined by the relative variation between the particular curves at each energy, hence it is obvious that the two regions are not visible due to the scaling to see the whole spectrum. In contrast, the region between the two maxima has almost none visibility, because this region lies in between the fractional Talbot distances, which results in no intensity pattern.      


\section{Conclusion}
First of all it is important, that the measurements and results presented above have a more qualitative than a quantitative nature, because the detector material dependent influences as 'escape events' or 'low energy events' of course induced errors in the source spectrum. Thereby the influence is especially tremendous for the results of the first two sections \ref{subsec:spectra} and \ref{subsec:gratspec}, because there the spectra itself are compared. In contrast the results of the energy-resolved stepping and the corresponding energy-visibility-map have a more quantitative issue, because here the 'escape' and 'low-energy events' can be assumed as intrinsic property of the spectra itself and hence only contribute onto the result due to statistical variations.\\ 
Nevertheless, in general the efficiency of the setups standard detector is almost the same up to energies of $80\,$keV, which is important to know for pure absorption measurements, because each material needs another X-ray energy for a proper image cf. Figure \ref{spectra6080}.\\
Additionally, the gratings have a strong influence on the intensity as well as they flatten the spectrum to almost the same intensity at the different energies, see Figure \ref{specgrat}. Here, the mean energy of the spectrum at an maximum energy of $60\,$keV is almost $45\,$keV, which perfectly hits the design energy of the Interferometer.\\ 
The analysis of the energy resolved phase-stepping and the corresponding energy-visibility-map in Figure \ref{stepvisi} is also in good accordance, as the maximum visibility occurs around an energy of $\approx 42\,$keV. The arise of the second maximum visibility around $13\,$keV is as mentioned possibly caused by combination of several effects and has to be investigated in more detail to give a statement, if the Talbot-effect is the main reason of this feature. 
%- shortly summarize results\\
%- change due to max energy -> adjust energy for measurements for diffe samples\\
%- phase stepping  induces biggest variation around design energy which is good\\
%- but also quite good visi at low E\\
%- maybe induced by absorption edge?? ask\\