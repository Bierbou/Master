\chapter[SSR characterization]{Characterization of the spatial system response}\label{chap:sysresp}
In this chapter, the general response of the system in the spatial as well as in the frequency domain is characterized, whereat special attention is paid to the determination of the spot-size of the source in different directions. It is important to know this property of an imaging system, because with this knowledge the images can be corrected afterwards e.g. for blurring due to the extent of the source size. In favour, two different techniques are presented in this chapter to obtain this property. The approach in the first part of the chapter completely relies on the spatial domain, whereupon in contrast the second technique makes a detour across the frequency domain, but of course with the Fourier transform the transition between these two spaces is very convenient. There are several methods to determine the spot size of the source e.g. with thin slits, rods or pin-holes. For a detailed overview about the different methods and corresponding norms see: \citep[see: Bavendiek et. al 2012]{Bavendiek2012} or \acrshort{en},\acrshort{astm},\acrshort{iec}.     
\section{Knife edge measurements}\label{sec:kedgemeasurements}
In this section the acquisition of the data and the resulting outcome for the measurements with a knife-edge is presented. At the end of the chapter a comparison between this technique and the second approach using a resolution-target, presented in \ref{sec:targetmeasurements}, is given. 
\subsection{Data acquisition}\label{subsec:knifedata}
This section focusses only on the spatial domain, because the main interest lies on the determination of the source size in different directions. The most common way to determine this property is to use a sharp edge, which is projected onto the detector screen. This projected edge-profile is also known as the \gls{esf} described in section \ref{subsec:esflsf}.   
An image of such a sharp edge and their corresponding projection is shown in Figure \ref{edgeimages}. a) shows the projection image of the edge, which was taken at a photon energy of $60\,$keV, a power of $80\,$W and with a magnification of $24.45$. The edge is on purpose tilted around an angle of about $3\,$° with respect to the detector pixel, to achieve a sub-pixel resolution in the later data-processing algorithm. As one can clearly see, the edge is smeared out over several pixels, due to the geometrical unsharpness induced by the limited spot size. b) shows the edge which has a polished side to smooth the surface and hence improve the sharpness of the edge.

\begin{figure}%[h]
	\begin{center}
		\includegraphics[width = 14.7cm,keepaspectratio =true]{edgeimagesgauss}
	\end{center}
	\caption[Pictures of the knife edge and corresponding projection image]{\textit{a) Image of a projected edge acquired by the Paxscan detector. The edge is slightly tilted ($\approx 3\,$°) to provide afterwards a sub-pixel resolution at the data-processing.  b) Two pictures of the cuboid with the edge profile on two opposite sides. The two thin sides of the cuboid are polished, to get a smoother surface and thus a sharper edge. c) Projection of the blurred edge projected onto the plane perpendicular to the edge direction and fit of the \gls{esf} with an error-function. d) Gaussian-fit with the acquired parameters of the error-fit. e) Illustration of the variation of the \gls{fwhm} for different angles between beam-axis and the edge. The vertex of the parabola indicates the best angle aligning the edge perpendicular with respect to the beam direction.}}
	\label{edgeimages}
\end{figure}
\clearpage 
To reinforce the contrast in the projection image, the cuboid containing the edge has a thickness of $5\,$mm and is made of stainless steel, because of its good absorption properties up to high photon-energies and the possibility to easily polish the surface. Usually the thickness of such edges is thinner, because the thinner the edge the less the influence of a misalignment of the edge onto the measurement. But as shown in section \ref{subsec:weth} the influence of the edge-thickness is negligible over a broad range, given that the edge is positioned perpendicular to the beam.
\paragraph{Edge alignment:} \label{alignment}
To fulfil this condition the edge is positioned on the Euler-cradle and first aligned by hand to be perpendicular to the beam and parallel to the detector pixel. After that the centre of the beam on the detector screen is determined. This is helpful for the subsequent measurement, because the placement of the edge right in the centre of the beam yields a edge image, which stays at the same position at the detector plane, regardless of the distance to the source. At any other position, the edge will be shifted across the detector plane while changing the distance between source and edge, due to the cone-beam geometry of the source. To adjust the edge as perpendicular to the beam as possible, the edge is positioned in the centre of the beam at a distance of $\approx 70\,$cm. After that, the edge is rotated in small steps in both directions around its own vertical-axis, to misalign the edge on purpose. At each of these steps an image is taken, whereupon each different angle between edge and beam-axis results in a more or less broadened edge image. At angles close to 90 degrees between beam-axis and edge, the projection of the edge is less broadened than at angles far away from 90 degree. This means, there is a minimum of the broadening of the edge.\\ 
Afterwards, each projection is corrected with a \gls{flat}. Subsequently, the projected images comparable to Figure \ref{edgeimages} a), are projected onto the plane perpendicular to the edge, to improve the statistic and reduce the image noise. This procedure is comparable with a line-plot over each pixel row and a subsequent averaging over all pixel rows. This projection of the edge shows the intensity variation between the different pixels. Such a projection of the edge-profile or the corresponding \gls{esf} is shown in Figure \ref{edgeimages} c). The shape of the profile reminds of a blurred step function, at which the shape can be roughly split in three parts. One part which has none intensity, because inside the cuboid the X-rays are almost completely absorbed. Another part with high intensity in each pixel, because no absorbing material is inside the beam. These two parts lead to two horizontal lines, because the neighbouring pixel have the same intensity values. Finally the third part right in between the other two parts, which is the projection of the edge itself. There, the shape of the curve is an increasing or decreasing line, with a slope correlated to the geometrical unsharpness and the misalignment of the edge. On this shape an error-function is fitted to get parameters for the comparison of the different \glspl{esf}.\\
Since the steepest \gls{esf} corresponds to the angle which is closest to a angle of $90\,$° between beam-axis and edge, the slope of the fit-functions is plotted against the different angles. This results in a parabola, whereby the vertex of the parabola indicates the best perpendicular position of the edge, see Figure \ref{edgeimages} d). If the edge was well aligned in the first place the value of the best angle should be near the start value. This procedure is at first done in rough steps the start angle and afterwards with finer steps around the best value of the first measurement. 
\paragraph{Measurement procedure:}
After these steps, the edge is properly aligned to start with the measurement of the system response. Therefore the edge is measured at three different distances with respect to the source. Usually one distance is sufficient for the determination of the \gls{psf} of the system and the corresponding spot size, but with three different distances the exploration of the influence of the magnification as well as the influence of the detector onto the measurement results is possible. The different distances are chosen to be close to the source plane to reduce the influence of the detector \gls{psf} and also to get a sufficient magnification of the \gls{esf} of the edge. The high magnification is necessary, because otherwise  the determination neither of the \gls{esf} nor of the \gls{psf} is possible, since the pixels and the \acrshort{psf} of the flat-panel detector are to big. The different distances and the cuboid dimensions for the measurement are exhibited in Figure \ref{edgedist}. As illustrated by the small coordinate system right in front of the detector-screen, the beam-axis is in the further set equivalent to the z-, the horizontal-axis to the x- and the vertical-axis to the y-axis, respectively.
\begin{figure}[h]
	\begin{center}
		\includegraphics[width = 14.7cm,keepaspectratio =true]{edgedist}
	\end{center}
	\caption[Illustration of the different edge positions for the measurement]{\textit{Illustration of the different edge positions during the measurement. Due to magnification the projection of the edge is spread over the detector plain perpendicular to the edge direction. The edge is placed on the Euler cradle, which is for reasons of clearness not shown here. To get an easier insight of the different arrangements a small coordinate system is placed right in front of the detector screen, thereby x- and y-axis are equivalent to the horizontal and vertical direction and the z-axis is congruent to the beam-axis of the X-ray source.}}
	\label{edgedist}
\end{figure}
\clearpage 
%The distance between the different positions is equidistantly set to $15\,$cm, at one hand for reasons of simplicity and at the other because the Euler-cradle has also a limited traverse paths in every direction. 
The edge is at first slightly tilted around $\approx 3\,$° round the z-axes (cf. Figure \ref{edgeimages} a)), to provide a better resolution for the projection of the edge-profile. Due to the slight tilt, also any effects having the edge in between two neighboured pixel-lines can be avoided, as it can happen in the case of a perfectly horizontal or vertical aligned edge.  At each of the particular z-positions a \gls{flat} is taken. After the flat-field the edge is placed at each z-position in three different directions into the beam, one vertical, the second horizontal and the third at the bisecting line of the first two positions with respect to the beam-axis. 
%After a short waiting time, to decrease vibrations induced by the driving around of the Euler-cradle, an image at each particular position is taken. 
With this approach it is possible to determine the shape of the spot-size and also the rotation of the source-shape in the x-y-plane at once. 
%Thanks to the possibility to control every stage in the setup from outside, only the adjustment on the sample holder of the Euler-cradle is necessary and the measurement itself is done automatically using a control script, containing the different positions and parameter. 
This measurement is done for three different energies at $40,\ 60$ and $80\,$kV and at each of these energies repeated for different power-steps, to cover the whole possible power range of the respective energy. This implies e.g. for a photon energy of $60\,$kVp a range of $160\,$W, leading to over $50$ measurements, whereby every measurement contains $9$ projection images and three \glspl{flat} (one for each z-position).  With this images at hand, the determination of the \gls{psf} of the system is possible, but for the determination of the source size an additional measurement is needed.\\

To evaluate the source size, also the \gls{psf} of the Detector has to be determined. As mentioned in section \ref{subsec:complpsf} the results of the described measurement above are the convolution of the respective source and detector \glspl{psf}. The approach is very similar to the upper case, but much less time extensive. To determine the \gls{psf} of the detector, the same edge is placed right in front of the detector plane. With this it is made sure that the influence of the source is negligible, because the \gls{psf} of the source scales with a factor of $M-1$ and for the position of the edge right in front of the detector the magnification is in good approximation unity. The edge is placed again in vertical as well as in horizontal direction, but due to the lack of a feasible sample holder a measurement diagonally to the pixel was not possible. But this problem can be easily solved during the processing by quadratic addition of the vertical and horizontal value to get a measure for the diagonal \gls{psf} as explained in the next section. This measurement is also repeated for the different energies, because the \gls{psf} of the detector can not necessarily be expected to be constant for different energies. 
\subsection{Data processing}\label{subsec:edgeprocessing}
For reasons of simplicity, this section just considers the processing of a data-set of one energy, in this case the one for $60\,$kV, because the treatment of the data-sets is the same for every energy, disregarding some changes in the projection angle or some computing parameters. To get the information out of the projected edge-profiles, the images are at first corrected with the corresponding \gls{flat}. After that a rectangular section of about $250\times250$ pixel, which only contains the blurred edge, is cut out of the images to reduce computing time.
\begin{figure}[h]
	\begin{center}
		\includegraphics[width = 12.7 cm,keepaspectratio = true]{projanglerot}
	\end{center}
	\caption[Illustration of the projection of an edge profile ]{\textit{Illustration of the projection of a tilted edge onto a plane perpendicular to the edge direction. Each pixel is projected under the same angle onto the projection plane. With this technique the two dimensional edge image is reduced to a one-dimensional trace known as the \gls{esf}. Due to the tilt of the edge, every imaginable way covering a detector pixel, ranging from 0 \% (no coverage) up to 100 \% (full coverage), is achieved, and thus provides a sub-pixel resolution in the projection plane. adapted from \citep{Samei1998}.}}
	\label{edgeangle}
\end{figure}
This section is then projected perpendicular to the edge direction (cf. Figure \ref{edgeangle}), to get a \gls{esf} with a good statistic. To assure that the projection is accurately done to the plane perpendicular to the edge, the same procedure as in section \ref{subsec:knifedata} (edge alignment) is followed, with the difference that here the projector projects the edge on purpose under a wrong projection angle, and thus onto a plane not perfectly perpendicular to the edge. Afterwards The FWHM is again minimized to find the right projection angle. For this determination one image of the whole data set with a vertical oriented edge is sufficient, because the angles for the other projection directions are correlated to each other due to the Euler cradle and can be adjusted by simply adding $90\,$° for the horizontal case or $45$° for the diagonal case, to the projector angle. For this data-set an image of a nearly vertical edge at the second measurement position ($z_{2}$) with a power of $80\,$Watt is chosen. With the projection angle found by this analysis, the whole data-set is projected afterwards. As mentioned above, the edge is slightly tilted from the usual coordinate axes. Using this treatment it is possible to get a sub-pixel resolution in the projected image, because the projector can split in this case each pixel by a certain number of sub-pixels and thus enhance resolution of the \gls{esf}. The splitting of one pixel in several sub-pixel is just accurately possible with a tilted edge, because every possible shadowing setting of a detector pixel occurs. In contrast to that, for an edge aligned exactly vertical to the pixel rows only three possible shadowing settings occur, either the edge covers the whole pixel, the edge do not cover the pixel, or the edge is somewhere in the middle of the pixel.\\
\begin{figure}[h]
	\begin{center}
		\includegraphics[width = 13.29 cm,keepaspectratio = true]{errorgauss}
	\end{center}
	\caption[Fit of the projection of the projected edge and corresponding Gauss-fits]{\textit{Illustration of The projected edge-profile with the associated fitted error-function, and the resulting Gauss-fits at an energy of $60\,$kVp and a power of $80\,$W. Top: Plot of three projected edges and their corresponding error-function fit for three different magnifications, respectively. To simplify the fitting, the data are normed. Bottom: Plot of the Gaussian functions fitted with the acquired fit parameter of the error-functions above, but rescaled to the determined values. The \gls{fwhm} of these fits is needed to determine the spot-size of the source.}}
	\label{errorgauss}
\end{figure}


To facilitate the further steps the \gls{esf} is normed between zero and one. On this normed \gls{esf} a error-function is fitted. As shown in section \ref{subsec:esflsf}, the \gls{lsf} is the first derivative of the \gls{esf} and in this case the first derivative of an error-function results in a Gaussian-function. For that reason the parameter of the error-functions can be taken to fit the \gls{psf} of the system. To get a better imagination what happens at the particular processing steps Figure \ref{errorgauss} shows some results for the error-function fit and the Gaussian fit. The plots show the results for the case of a vertical edge at an energy of $60\,$kVp and a power of $80\,$W, for the different edge distances. The top plot shows the \gls{esf} with the corresponding fit of the error-function, and the plot below the related Gaussian fits. As one can clearly see, the influence of the magnification onto the width is tremendous, but after correction with the respective factor the tree curves should have the same width. As suggested in the latter section for the determination of the source size also the \gls{psf} of the detector is needed. The extraction of the \gls{psf} out of the raw images is the same as for the other measurements and is for that reason not explained in further detail. Only for the detector PSF in the diagonal direction and their corresponding \gls{fwhm} an additional step is needed. As the adjustment of the edge diagonal to the pixel was not possible, the PSF for this direction is approximated using simple mathematical and geometrical considerations. Thereby it is assumed, that the vertical and horizontal \gls{psf} are the prime-axes of the two-dimensional detector \gls{psf}. The diagonal detector PSF is determined by quadratic multiplication of the vertical and horizontal detector \gls{psf}.   With this at hand the spot-size of the source can be determined for the three directions. Therefore the equation \ref{spotwidth} of section \ref{subsec:complpsf} is rearranged to:
\begin{equation}
\sigma_{source} = \frac{\sqrt{\sigma_{system}^{2}-\sigma_{detector}^{2}}}{M-1},
\end{equation}
and afterwards multiplied by $2\sqrt{2 ln2}$, which finally yields the \gls{fwhmg} of the PSF of the source spot in one direction and for one distinct power:
\begin{equation}\label{sourcedet}
FWHM_{source} = \frac{\sqrt{FWHM_{system}^{2}-FWHM_{detector}^{2}}}{M-1},
\end{equation}
whereby $M$ denotes the different magnification factors for the different distances of the source. With this equation the determination of the spot-size in the three different directions is possible. In the next section the results for the different measurements are discussed in further detail.      
\subsection{Results and discussion}\label{subsec:edgeresults}
In this section the results of the various measurements at different positions are presented and discussed. The motivation of these measurements was the determination of the spot-size of the source, and thus the entire resolution of the setup. At first the measurement of the detector \gls{psf} is presented. As mentioned above, the \gls{psf} of the detector is needed to determine the \gls{psf} of the source from the convolved \gls{psf} of the system.
\begin{figure}[t]
	\begin{center}
		\includegraphics[width= 14.7 cm,keepaspectratio = true]{detpsf60kv}
	\end{center}
	\caption[Dependency of the detector PSF]{\textit{Dependency of the detector PSF for the vertical and horizontal direction. For better statistics the measurement was done twice, the first one at 09.04.2015 indicated by the green line and once again at 13.10.2015 indicated by the blue line. a) Vertical PSF of the detector for different power of the source. b ) Horizontal PSF of the detector for the same power steps of the source. For reasons of clarity, only the mean PSF of the blue line is shown in both plots. Both plots show the same dependency. The PSF is in both cases up to $100\,$W nearly horizontal, except the green line in b). A main feature is the difference of $\approx 40\, \mu$m between the values of the horizontal and vertical PSF, which is possibly induced by the different source size in the horizontal and vertical direction. The measured values differ strongly from the denoted PSF of $\approx 254\, \mu$m by \citep{Paxscan}, except the values at $150\,$W.}}
	\label{detpsf60}
\end{figure}
\paragraph{Detector PSF:}
According to the data sheet of the Varian PaxScan detector the detectors \gls{psf} is specified to $\approx 254\, \mu$m. During the measurements presented above the \gls{psf} of the detector was determined twice for three different energies, $40,\ 60$ and $80\,$kVp, in vertical and horizontal direction, to specify the behaviour for different energies. Furthermore the power was also varied at $60\,$kVp, to compare this behaviour as well to the reference value of the data sheet.
\begin{table}[h] 
	\begin{center}	
		\begin{tabular}{c|c|c|c|c}
			\multicolumn{2}{c|}{direction} & $40\,$kVp & $60\,$kVp & $80\,$kVp \\ \hline \hline
			\multirow{2}{18 mm}{vertical PSF [\text{$\mu$m}] }\rule{0pt}{13pt}  & October & 348.6 & 381.3 & 279.2 \\ \cline{2-5} 
			\rule{0pt}{13pt}	& April & 349.5& 386.5 & 281.9 \\ \hline
			\multirow{2}{18 mm}{horizontal PSF [\text{$\mu$m}]}\rule{0pt}{13pt} & October & 354.4 & 423.2 & 367.1 \\	\cline{2-5}											\rule{0pt}{13pt} & April & 361.7& 451.8 & 396.4 \\ \hline
			\multirow{2}{18 mm}{diagonal PSF [\text{$\mu$m}]}\rule{0pt}{13pt} & October & 351.5 & 402.8 & 326.1 \\	\cline{2-5}											\rule{0pt}{13pt} & April & 355.6 & 420.4 & 343.9 \\
		\end{tabular}
	\end{center}
	\caption[Comparison of the different detector PSFs]{\textit{Results of the measurements of the detector PSF for different energies and directions. There is merely a marginal difference between the values of the first and the second measurement denoted by "April" and "Ocotober" accordingly to the green and blue line in Figure \ref{detpsf60}, which confirms the correctness of the measurement. The difference between the vertical and horizontal direction increases constantly from $40$ to $80\,$kVp. The values for the diagonal direction were calculated by quadratic addition of the vertical and horizontal PSF values.}}
	\label{detpsfs}
\end{table}
The results for the measurement at $60\,$kVp are shown in Figure \ref{detpsf60}. Here, a) shows the \glspl{psf} for the vertical direction and b) the \glspl{psf} for the horizontal direction. The two different measurement are indicated by the 'blue' and 'green' colour, whereupon the "green" line shows the first measurement and the "blue" line the repetition of the same measurement about half an year later. As one can see, the behaviour for the different powers is in both plots up to a power of $100\,$W almost a horizontal line. The only exception is the green line in b), which also has fluctuations before $100\,$W. After that value the \gls{psf} drops down for all measurements. A possible explanation for this is the saturation of the detector above this value. Nevertheless, the mean detector \gls{psf} averaged over the data points of the blue lines before $100\,$W is in the vertical as well as in the horizontal case very close to the corresponding values. The detector \gls{psf} of the other energies was only measured at distinct power, for $40\,$kVp at $50$ and $60\,$W, and at $80\,$kVp for both measurements at $100\,$W. The results of the respective measurements are presented in Table \ref{detpsfs}, whereby for the measurement at $60\,$kVp the averaged value is shown to have a better comparison between the different energies. The values for the diagonal \gls{psf} were calculated by quadratic addition of the values of the vertical and horizontal \glspl{psf} comparable to the calculation of the diagonal of a rectangle. One main result of theses measurements is, that the horizontal PSF is always bigger compared to the vertical PSF. One possible explanation for this is the behaviour of the source, because as illustrated in the further section, the horizontal source size is always bigger than the vertical source size. This has maybe influenced the edge-profile on the detector screen and induced this difference. This explanation is also in good agreement with the presented values, because the difference of the \glspl{psf} of the different directions increases with the energy and the power, just as the source size. In the next part these results are used to determine the spot-size of the source from the measured PSF of the entire system.    	

\paragraph{Spot-size of the source:}
As mentioned above, the spot-size of the source is defined in this thesis as the \gls{fwhmg} of the PSF of the source. The FWHM can be determined with equation \ref{sourcedet}. For convenience, the results of the spot-size measurements are exemplary discussed on the data set for an energy of $60\,$kVp, because the treatment of the results for $40$ and $80\,$kVp can be explained analogical. The results for the sizes of the source in different directions are shown in Figure \ref{60kvpcombi}, whereby a), b) and c) illustrate the vertical, horizontal and diagonal spot-size of the source, respectively. The different distances between source and edge are colour-coded in the respective plots. To discover the size of the source in the source-plane, the curves are corrected with the respective magnification factor of the different distances. Under perfect conditions, the different lines should completely coincide, but due to variations during the measurements the lines are slightly separated from each other. 
\begin{figure}%[h]
	\begin{center}
		\includegraphics[width= 14.7 cm,keepaspectratio = true]{60kvpcombi}
	\end{center}
	\caption[Results of the measurement of the source size for different directions ]{\textit{Results for the determination of the source size for different edge-directions and magnifications. a) Results for the vertical spot-size of the source. b) Results for the horizontal spot-size. c) Results for the diagonal spot-size. The different colours in the plots indicate the different magnification factors, whereby the dashed line indicates the mean value of theses three different magnification. A conspicuous feature is the congruence of the mean values with the curve of the second position of the edge, here denoted as green line.}}
	\label{60kvpcombi}
\end{figure}
\clearpage
Nevertheless, the mean spot-size of the different directions fits merely perfect the result for the edge at the position $z_{2}$ see Figure \ref{edgedist}. The shape of the plots is nearly for all cases the same, regardless of the edge direction, and can be separated in two parts. 

The first part ranges from $0\,$W up to about $30\,$W and the second from $30\,$W up to the end of the measurement. However, the results below $30\,$W are subject to large errors, because the spot-size of the source is in this region very narrow, as further explained in section \ref{sec:targetmeasurements}. Due to this fact, the resolution of the spot in this region is not possible with a pixel size of the detector of $127\,\mu$m, because even with a magnification of $24.45$, e.g. a spot-size of $5\,\mu$m lies within one pixel and is therefore not resolvable, regardless of a sub-pixel resolution in the algorithm or not.\\

Additionally, the intrinsic PSF of the detector itself hinders the exact determination of the spot-size, because even for bigger spot-sizes the projection of the edge is blurred due to the \gls{psf} of the detector, which leads to problems properly deconvolving the particular parts of the systems \gls{psf}. 
The only statement which can be delivered from this is, that the spot-size of the source is nearly constant, within the region below the defocussing point of the electron-optic at $25\,$W. For that reason, this region is further investigated in section \ref{sec:targetmeasurements}. The second part can be split in two sub-regions with nearly linear behaviour, the first starts at $30\,$W and ends at $50\,$W, the second starts subsequent at $50\,$W up to the end of the power range. This shape is induced by the defocussing of the electron-optic avoiding a damage of the target. As mentioned earlier, the electron-optic abruptly diminishes the focussing of the electron beam at $25\,$W. After that the spot-size rises enormous during the next few power steps, until the optic has reach the next defined focussing point. This region can be compared to the part of a square-root function for values below unity, where the function has a very steep slope. After this the focussing of the optic decreases more constant, which yields a much slower increase of the source diameter and thus a slower increase of the spot-size itself. Nevertheless, there are two results striking the eye at first glance.
\begin{figure}
	\begin{center}
		\includegraphics[width= 14.7 cm,keepaspectratio = true]{60kvpcomparison}
	\end{center}
	\caption[Comparison of the two spot-size measurements at 60 kVp]{\textit{Comparison of the two measurements of the spot-size of the source at an energy of 60 kVp. The green line denotes the first measurement of the spot-size, the blue line indicates the revision measurement and is hence called 'control' measurement. a)-c) show again the spot-sizes for the different directions in the same order as in Figure \ref{60kvpcombi}. The intrusion between $100$ and $125\,$W disappeared in the 'control' measurement and the values increased in every direction, but the general shape of the curves did not change. The changes between the two measurements are induced by the source, e.g. the filament changed of the target was rotated between the two measurements.}}
	\label{60kvpcomp}
\end{figure}
The first one, the horizontal spot-size is always much bigger than the vertical spot-size (compare Figure \ref{60kvpcombi} a) and b)). The second one, each plot shows the same  intrusion at the same power values from $100$ to $125\,$W. To explore this feature in more detail, a second measurement was done with the same parameters. The results of this measurement are shown in Figure \ref{60kvpcomp}. For better comparison, only the mean values of both measurements are shown in the plots for the different directions of the spot-size. The second measurement is denoted by the blue line. Obviously, the shape of the curve did not change, but the intrusion between $100$ and $125\,$W disappeared and the values are bigger for all measured directions. The explanation for these variations between these two measurements are intrinsic properties of the source itself. One reason for example is the slight rotation of the source target after a certain amount of time. This is necessary, because during the operation of the source, the electrons impact always at the same position of the target. Hence, very small amounts of target material are dissipated at this position, which destroys sooner or later the small size of the spot. This process is of course very slow, comparable to the dissipation of material by a river.
\clearpage
Therefore it is possible, that one measurement took place right before and one after the rotation of the target. Additional, the target material is not perfectly homogeneous over the whole surface and thus the interaction of the electrons within the targets surface can differ.\\

At the second glance, the decrease of the curves for high powers strikes the eye. This behaviour can be observed again for every measurement presented in the Figures \ref{60kvpcombi} and \ref{60kvpcomp}, besides for the case of the 'control' measurement for the diagonal spot-size. Anyway the behaviour of this curve can not be explained properly. In contrast to this, a possible reason for the decrease of the other curves for high powers is again the saturation of the detector at high intensities, as explained for the detector PSF, which diminishes the spread of the edge over the detector screen. In general, the spot-size of the source is always in the micro-metre range, but true micro-focus properties are only reached below the defocussing point at $25\,$W of the electron-optic. So for a measurement which requires high resolution a power below $25\,$W should be chosen, because in this region a very good resolution far below $100\, \mu$m can be assured. To get a clue of the spot-size of the source for different energies, the same measurement was performed at $40$ and $80\,$kVp. The results of theses measurements are depicted in Figure \ref{4080kvpcombi}, whereby the plots a)-c) show the spot-sizes at an energy of $40\,$kVp and the plots d)-f) the spot-sizes at an energy of $80\,$kVp for the different directions, respectively. The general behaviour of the different curves is  similar to the curves at $60\,$kVp, except for the horizontal and diagonal direction at $80\,$kVp. The spot-sizes decrease for this cases after a power of about $80\,$W and stay nearly a the same size above $\approx 120\,$W. This behaviour can not be completely explained, but the reason could be again the saturation of the detector, because the saturation starts at an earlier point for higher energies.                  	
\begin{figure}
	\begin{center}
		\includegraphics[width= 14.7 cm,keepaspectratio = true]{4080kvpcombi}
	\end{center}
	\caption[Results for the spot-size measurements at an energy of $40$ and $80\,$kVp]{\textit{Results of the spot-size measurements for an energy of 40 and 80 kVp. a) -c) : Results for the spot-sizes in the different directions at 40 kVp. d)-f): Results for the spot-sizes in the different directions at 80 kVp. }}
	\label{4080kvpcombi}
\end{figure}
\clearpage
\paragraph{Conclusion:} In this section, the results for the various measurements of the spot-size of the source in different directions and energies, presented in the previous section, are summarized. 
First of all it can be stated that the general behaviour of the curves for the spot-sizes in different directions do not change much regardless of the different energies or directions. The vertical size of the spot is always much smaller than the horizontal size. This information is for example very useful, if one has to decide in which direction the gratings should be orientated because the aim is often to enhance the final resolution of the image, which is directly proportional to the size of the spot. Thus it is even reasonable using a grating interferometer to take care putting the grating lines perpendicular to the direction with the smaller spot-size, because this also enhances the resolution of the fringe-pattern. Secondly the source looses its micro-focus properties after a applied power of $25\,$W to prevent damage of the target. Another important result is the fact, that the size of the source strongly depends on the state of the different parts e.g. filament and target, of the source and can therefore vary for each measurement as illustrated in Figure \ref{60kvpcomp}. For that reasons a measurement of the current source size is recommended before the main measurement, if a proper deconvolution of the results of the measurement is needed. Due to problems with the too big detector pixels it was not possible to determine the spot-size quantitatively below about $25\,$W. Therefore as mentioned above, another approach with the aid of a resolution target is made in section \ref{sec:targetmeasurements}. \clearpage        


\subsection{Shape of the source-spot}\label{subsec:spotshape}
In a next step the shape of the source spot is approximated with the results of the spot-sizes in different directions. As a first approach, an ellipse-fit is used to describe the source shape, because due to geometrical considerations of the source and thus the geometry of the electron-beam and the resulting X-ray beam produced by the e-beam hitting upon the reflection-target this shape is predicted as one can see in Figure \ref{sourceellipse}. Usually, the determination of two directions perpendicular to each other is sufficient to determine an ellipse, because an ellipse can be determined by its two half axes. Nevertheless, an ellipse has also another degree of freedom, the rotation around the centre. Therefore, a third measurement direction is needed. Since a priori no knowledge about the shape or the rotation of the source spot was available, the measurements accounted all degrees of freedom for the assumed ellipsoidal shape of the spot. If the rotation of the ellipse around the centre is not accounted, the determination is much easier, because the two prime-axes of the ellipse have not to be determined, but are given by the half of the perpendicularly determined spot-sizes. The half of these two values can be used as so called 'big-' and 'small-' prime-axis of the ellipse.
\begin{figure}[h]
	\begin{center}
		\includegraphics[width= 14.7 cm,keepaspectratio = true]{spotshape}
	\end{center}
	\caption[Illustration of the geometry of the electron-beam and the resulting X-ray beam of the source]{\textit{Drawing of the geometry of the electron-beam and the resulting geometry of the X-ray beam, generated by the electrons hitting upon the reflection target. The graphic shows the geometry under different angels of view. The shape of the electron beam is squeezed in vertical direction dependent on the tilt-angle of the reflection-target and thus yields an ellipse shaped X-ray beam. adapted from: X-RAY WorkX GmbH}}
	\label{sourceellipse}
\end{figure} 
\clearpage
%\paragraph{Data transformation in x-y coordinates.} 
With this one can easily calculate the ellipse with the help of the parametric representation for the x and y direction \citep{Mathematik1997}: 
\begin{equation}\label{paramell}
		\begin{pmatrix} x(t)\\ y(t) \end{pmatrix} = 
		\begin{pmatrix} x_{0} \\ y_{0} \end{pmatrix} +
		\begin{pmatrix} \cos(t)\cos(\phi)-\sin(t)\sin(\phi)\\ \cos(t)\sin(\phi) + \sin(t)\cos(\phi)\end{pmatrix} \cdot
		\begin{pmatrix} a\\b \end{pmatrix}.
\end{equation}
Here, $t$ ranges from $0$ to $2\pi$ and is the parameter to calculate the different ellipse points. $a$ and $b$ are the two prime-axes, $x_{0}$ and $y_{0}$ are the origin of the ellipse and $\phi$ is the ellipses rotation angle, which is for this simple case set to zero.\\

To  calculate all degrees of freedom of the ellipse, the procedure is more complicated. Therefore, the measured values of the different spot-sizes have to be at first transformed into the x-y coordinate system, to be able to determine the correct prime-axes of the ellipse and their rotation angle $\phi$ with respect to the x-axis of the coordinate system. The transformation is again only shown for the measurements at $60\,$kVp as representative for all other measurements. The measured values of the spot-sizes are given in polar coordinates, at which in this representation the values for the radii are just the determined spot-sizes and the respective angles are given by the tilt of the edge.
\begin{figure}[h]
	\begin{center}
		\includegraphics[width = 14.7cm,keepaspectratio = true]{ellipsexy60kvpcombinogrid}
	\end{center}
	\caption[Transformed spot-size into x-y coordinates]{\textit{Representation of the measured spot-sizes for the different directions in in x-y coordinates. a) and b) look very similar in this representation. The reason for this is the change of the coordinate system from polar to Cartesian coordinates which doubles the data-points of each power and additionally makes the separation of the data-point of the particular powers almost impossible.}}
	\label{xycoordsnogrid}
\end{figure}
\clearpage 
Hence they can be easily transformed into a Cartesian coordinate system. The transformed values for the different spot-sizes at a photon energy of $60\,$kVp are depicted in Figure \ref{xycoordsnogrid}. Each spot-size yields two points with contrary x-y values, because the measured size is the diameter and thus twice the radius. This provides with three different directions 6 data-points for each power. a) illustrates the transformed values for the first measurement at $60\,$kVp and b) the values for the second measurement. a) and b) look very similar in this representation due to the doubling of data-points caused by the transformation, why the separation of the values of the different powers is really hard.\\

The approximation of an ellipse onto these data is a linear minimization problem and can be solved for each respective power using a Lagrangian-multiplicator. The exact explanation of this technique can be found in \citep{Honerkamp1993}. After calculation of the prime-axes and the rotation angle for each power, the x and y coordinates for the particular ellipse points are determined with equation \ref{paramell}. With this, the shape of the source spot at each power step can be fitted. 
\begin{figure}[h]
	\begin{center}
		\includegraphics[width = 14.7cm,keepaspectratio = true]{ellipse60kvpselection}
	\end{center}
	\caption[Selection of ellipse-fits for an energy of 60kVp]{\textit{Selection of the ellipse-fits for the measurement at 60 kVp. a) shows the fits for the first measurement. b) the fits for the control measurement. The blue marker in both plots indicate the measured values of the different spot-sizes in the different directions. Both plots show the ellipse-fits for the same power-steps. The surface area of the ellipses in a) compared to b) is always smaller for the different powers and the the change of the source-shape at high powers is much less in a) than in b).}}
	\label{ell60kvpselec}
\end{figure}
%\clearpage
As the determination of the spot-size for low powers up to $25\,$W did not work well, these fits are not provided here. One additional problem which appeared, was the truncation of the fit-algorithm due to the values of the diagonal spot-size. Sometimes the diagonal values were to small compared to the corresponding values for the horizontal and vertical direction, which makes it impossible to fit a proper ellipse on this data subset, because to minimize the distance between the ellipse and the data points, the curvature has to be inwardly, which obviously destroys the ellipse. Therefore, the whole data set was fitted with the simple approach of equation \ref{paramell}, disregarding the rotational degree of freedom which in addition induces big errors to the fit algorithm due to the lack sufficient fitting-points. A selection of the fitted ellipses and their originally measured spot-sizes for the different directions are illustrated in Figure \ref{ell60kvpselec}. Here, a) and b) show the results for the two measurements at an energy of $60\,$kVp. a ) and b) show the ellipses for the same power-steps. Thereby, the ellipses for the power-steps below $30\,$W are neglected, because as mentioned above they have no significance about the spot-size. As the spot-size increases very strongly after $30\,$W and slows down after $50\,$W, all ellipses in between this region are shown to illustrate the increase of the area of the source. For the residual power range particular ellipses are selected in bigger intervals. Both plots show a homogeneous growth of the spot-size in the different directions, whereupon the ellipses in both plots have almost the same shape, but different values. These plots remarkably indicate the change of the area of the source-spot, but the shape stays very similar between different measurements.
%To get a feeling how the source size evolves in '3D', Figure \ref{3d} shows the ellipses of the particular power- steps between $30$ and $150\,$W in 3D-plots  at different angles, whereupon the particular ellipses lie in the x-y-plane and the z-axis equates the power-axis.    
%-3d plot maybe markus can help plotting\\
%-plot of projected ellipse and 3d plot\\
%-
%\clearpage
\section[R-target measurement]{Resolution-target measurement}\label{sec:targetmeasurements}
In this section another approach for the determination of the spot-size of the source is presented. Thereby, the focus especially lies on the determination of the spot size for small powers, which is not possible when using edge measurements. In this section, a \gls{restarget} is used for the determination of the source size. Such a target is usually used to determine the resolution of various imaging systems. The principle of such a target is quite simple: Patterns of lines of a high absorbing material with different distinct thickness and distances are arranged on a substrate of low absorbing material. Thereby, the thickness of two neighbouring lines and the distance between them increases in the same manner. The determination of the PSF with this method is a bit more complicated, because the PSF can not be determined directly, rather the \gls{mtf}, which has to be Fourier transformed to obtain the PSF.
\subsection{Data acquisition}\label{subsec:targetdata}
In contrast to the last section, the system response in this approach is firstly determined in the frequency domain, because with a \gls{restarget} one is primarily able to determine the contrast between the lines an the spaces and thus the \gls{mtf} of the system. 
%As one can see in Figure \ref{restargetcomp}, 
Such a target provides various types of resolution patterns. The resolution-pattern used for the measurement are parallel arranged lines in a long row with distinct distances and corresponding line-thickness, which can be seen in the outer region of the image. This pattern provides the possibility to measure the resolution starting form $32$ down to $4\, \mu$m. To avoid the influence of the detector \gls{psf}, the target is put very close behind the source, about $1\,$cm. This leads to a magnification of about 195. As a side effect of this high magnification, different images of the target for one power have to be taken, because only a small part of the whole target is illuminated per image. Due to difficulties of the target alignment of the diagonal direction for the different images, this approach focusses on the vertical and horizontal direction.\\

%\paragraph{Measurement procedure:}

The measurement procedure is analogue to the one described in the previous section, but has slight differences. One difference is, the lines of the target are aligned vertically and horizontally to the detector pixels without any tilt. Nevertheless, a small tilt can not be avoided, but the angle is very small $\approx 0.005$°, which is approximated by the ratio of the shift of pixels in x-direction of the straight border line of the resolution pattern over all pixels in y-direction. For that reason it can be assumed that the tilt of the resolution pattern has no influence on the later data processing. As for the edge measurement, a \gls{flat} is taken for each power and for each target-direction. After the flat-field image, the target is at first moved along the vertical-axis. Thereby, nine pictures are taken to cover the whole resolution target. Afterwards, this procedure is repeated for the horizontally orientated target. 
%Thereby, it is indispensable for the subsequent data-processing, that the particular pictures have a small overlap among each other. 

\begin{figure}[h]
	\begin{center}
		\includegraphics[width= 14.7 cm,keepaspectratio = true]{3w1pictarget}
	\end{center}
	\caption[First image of the resolution-target at $60\,$kVp and $3\,$W in vertical and horizontal direction, respectively.]{\textit{First images of the image-series for the vertical and horizontal direction at an energy of $60\,$kVp and $3\,$W. Both images are flat-field corrected and show the region between $4$ and $8\, \mu$m. a) clipping of the beginning of the resolution-target in vertical direction. b) section of the beginning of the resolution target in horizontal direction. Comparison of a) and b) yields that the resolution in a) is much better than in b), because the $4\, \mu$m lines in a) can be separated from each other in contrast to b).}}
	\label{restarget1pic}
\end{figure}
\clearpage
This measurement procedure is again repeated for the same energies as in the previous section. At each of these energies, the power is once again varied in small power-steps starting at $3$ up to $40\,$W to have the possibility to compare the measured spot-sizes with the corresponding values, obtained in the previous section. In contrast to the knife-edge measurement the position of the target between source and detector is not varied. To get a feeling how the particular images look like, Figure \ref{restarget1pic} a) shows the first image of the series for the vertical direction and b) the first image of the series for the horizontal direction, at an energy of $60\,$kVp and a power of $3\,$W. 
\subsection{Data processing}\label{subsec:targetprocessing}
As mentioned before, at least nine pictures are needed to cover the whole resolution target for one power, due to the large magnification. Therefore, the particular images have to be stitched together to obtain at once an image of the whole target. For that reason, each image is at first corrected with the respective \gls{flat}. Afterwards, the overlap between the particular images is determined, whereat the overlap is always the same between the different pictures, because the images were taken at equidistant steps, which simplifies the whole process. Due to this, the overlap has to be determined just once for the vertical and once for the horizontal direction. Subsequent, the overlapping pixels of two neighboured images are cut of and averaged, whereby this procedure is repeated for the remaining images. Then, the particular images and the related overlap between them are stitched together to a complete image of the resolution target one in vertical and the other in the horizontal direction. This procedure is then repeated for all power steps. Some results of the stitching algorithm are depicted in Figure \ref{restargetcombi}. a), c) and e) show the vertical \gls{restarget} for a power of $3$, $25$ and $30\,$W, respectively. b), d) and f) show the same target for the horizontal direction for the same power-steps rotated around $90$ degree to simplify the comparison between the particular images.
\begin{figure}[h]
	\begin{center}
		\includegraphics[width= 14.7 cm,keepaspectratio = true]{restargetcombicomp}
	\end{center}
	\caption[Images of resolution-targets for different power]{\textit{Images of the resolution-targets for different power at an energy of $60\,$kVp. From the left to the right the power of the source increases. a),c),e) show the resolution target in vertical direction for different power. b),d),f) show the resolution-target in horizontal direction for different power. The resolution of the vertical direction compared to the horizontal direction is better for every power. This confirms the results from section \ref{sec:kedgemeasurements}.}}
	\label{restargetcombi}
\end{figure}
Before the line-pattern can be projected onto the plane perpendicular to the lines, in the same manner as presented in the previous section, the obtained images are cropped so that they only contain the significant target lines and nothing additional. This results in alternating high an low values, whereupon high values arise at the pixels without any absorbing material and low values at the pixels covered by the target lines. Therefore, the negative logarithm of these values is taken to invert the values and thus obtain later the contrast in the right way.\\ 

To get a clue how the projection of the \glspl{restarget} look like the projected values are shown in Figure \ref{restargetproj} for two different energies and both directions. In this Figure a) and b) show the projected values for the vertical and horizontal direction at a power of $3\,$W and c) and d) the same resolution-target at a power of $30\,$W and an energy of $60\,$kVp, respectively. Thereby, the thickness of the target-lines and the particular spaces in between shrink form the left to the right, from $32$ down to $4\, \mu$m, which leads to a decrease of the amplitude between the individual values especially in the region close to the end of the resolution range.\\
 
In general the following rule applies: The earlier the amplitude decreases the worse is the resolution. To evaluate the contrast with these projected data another intermediate step is necessary. At this approach, the contrast between the lines and spaces is determined with the same formula as used for definition of the visibility in equation \ref{visibility}. Hence, all local maxima and minima, meaning the maximum an minimum intensity-values of the particular lines and spaces, have to be extracted from theses curves to obtain the so called Michelson-contrast. A main problem arises during the extraction-algorithm, due to the fact that the lines getting thinner and thinner and likewise the distance between the particular lines shrink. This fact makes it difficult to distinguish the individual maxima and minima and for higher power it gets impossible to obtain reasonable data. Therefore, the contrast between the different lines of the resolution-pattern was only determined up to a power of $30\,$W. As mentioned earlier, at higher power also the spot increases to much and hence each measurement in one particular direction is strongly influenced by the spot-size perpendicular to this direction. After the determination of the contrast of all lines of the resolution-pattern, the obtained values are normed to the maximum contrast, which arises at the largest line-thickness, meaning in this case $32\, \mu$m or expressed with spatial frequency at $31\,$lp/mm.\\

This scaling is necessary, because the \gls{mtf} which is equivalent to the determined Michelson-contrast is in general stated in percent. As the \gls{mtf} is a property in the frequency domain, the spatial resolution denoted at the target is converted to the equivalent spatial frequencies.
\begin{figure}[h]
	\begin{center}
		\includegraphics[width= 14.7 cm,keepaspectratio = true]{restargetprojcombi}
	\end{center}
	\caption[Illustration of the projected resolution-pattern for different power]{\textit{Illustration of the projected resolution-pattern at an energy of $60\,$kVp. a) and b): Projection of the vertical and horizontal resolution-pattern at $3\,$W. c) and d): Projection of the same pattern for the vertical and horizontal direction for $30\,$W. The sharp jumps between the particular local minima are artefacts of the stitching algorithm. As a rule of thumb holds: The earlier the amplitude decreases between the particular values the worse the resolution.}}
	\label{restargetproj}
\end{figure}
\clearpage
Usually the next step would be the Fourier transformation of the obtained \glspl{mtf}, which leads to the PSF of the source in the different directions and thus the spot-size, which can be compared with the obtained spot-sizes in  the previous section. Unfortunately, an easy Fourier transform was not possible, because the most obtained MTFs did not drop down to zero before the maximum spatial frequency, which can be resolved with the resolution-target. Therefore, it is not trivial to find a feasible fit-function. Even with a good initial guess of the fit-function, there are big errors induced during this step, which destroys the significance of the obtained values. to get a good transform a fit of the particular MTFs is needed. Nevertheless, there are several results and statements, which can be retrieved from the obtained curves of the MTFs for the different directions and powers.
\begin{figure}[h]
	\begin{center}
		\includegraphics[width= 14.7 cm,keepaspectratio = true]{MTFcombi}
	\end{center}
	\caption[Obtained MTFs for different power at an energy of $60\,$kVp]{\textit{Resulting MTF's of the resolution-target measurement at an energy of $60\,$kVp and different power. a) determined contrast values and associated smoothed MTF curves for the vertical target. b) determined contrast values and associated smoothed MTF curves for the horizontal target. The resolution drops down for increasing power, but in the vertical case the MTFs never drop down below 10 \% which is usually set to be the resolution-limit, besides for $25$ and $30\,$W, which is induced by the influence of the horizontal spot-size.}}
	\label{mtfcombi}
\end{figure}  
\subsection{Results and discussion}\label{subsec:targetresults}
In this section the outcome of the previous measurements is discussed. The results of the determination of the MTF in the previous part are illustrated in Figure \ref{mtfcombi}. Here, a) shows the obtained \glspl{mtf} for the vertical direction and different power at an energy of $60\,$kVp and b) the \glspl{mtf} for the horizontal direction. Both plots show the obtained values at each spatial frequency and the corresponding smoothed curve. The dashed line at 10 \% indicates in both plots the resolution limit. If the values of the MTF drop down below this value the resolution of higher spatial frequencies is not possible any more. The observed behaviour of the curves is in good agreement with the previous results. The \glspl{mtf} of the horizontal direction drop down much faster than the \glspl{mtf} of the vertical direction, which means that the resolution in horizontal direction is much worse than in vertical direction and thus implies, that the spot-size in horizontal direction is again much bigger than the spot-size of the vertical direction. As one can see in a), the resolution limit in vertical direction is never reached for a source power smaller than $25\,$W, which denotes that the spot-size in this direction is smaller than $4\, \mu$m, because also the finest lines of the \gls{restarget} can be distinguished.\\  

This is in good agreement with the first sighting of the observations made in Figure \ref{restarget1pic} a), at which the $4$ micron lines can still be visually distinguished, and also with the predicted source size of $2\, \mu$m by the data-sheet of the source cf. \citep{DatasheetX}. To get a feeling about the limited resolution of the different curves, the intersection points of the \glspl{mtf} with the 10 \% limit are depicted in Table \ref{table:reslimit}, whereupon the different values are expressed in units of \acrfull{sf} for the frequency-domain and also converted in $\mu$m for the spatial domain, which is referred to as PSF of the source.
\vspace{0.5cm}
\begin{table}[h] 
	\begin{center}	
		\begin{tabular}{c|c||c|c|c|c|c|c|c}
			\multicolumn{2}{c||}{Power [W]}	& 3& 5& 10& 15& 20& 25& 30\\ \hline \hline
			\multirow{2}{18 mm}{vertical resolution}\rule{0pt}{13pt} & Sf [lp/mm] & $>250$ & $>250$ & $>250$ & $>250$ & $>250$ & $ 235$& $200$ \\ \cline{2-9}
			\rule{0pt}{13pt} & spot-size [\text{$\mu$m]} & $<4$& $<4$ & $<4$ &$<4$& $<4$& $4.3$&$5$ \\ \hline
			\multirow{2}{18 mm}{horizontal resolution}\rule{0pt}{13pt} & Sf [lp/mm] & $ 192$ & $187$ &$175$ &$187$ &$189$ &$187$ &$73$ \\ \cline{2-9}
			\rule{0pt}{13pt} & spot-size [\text{$\mu$m]} &$5.2$ &$5.3$ &$5.7$ &$5.3$ &$5.3$ &$5.3$ &$13.6$  \\  	  
		\end{tabular}
		\caption[Resolution limits and correlated assumed spot-sizes]{\textit{Resolution limits of the different \glspl{mtf} obtained from the resolution-targets in vertical and horizontal direction and at different power. The assumed spot-sizes are registered below, whereupon these values are not the FWHM of the Fourier transform of the \glspl{mtf} but the converted spatial frequencies to have an approximative value.}}
		\label{table:reslimit}
	\end{center}    
\end{table}
For the \glspl{mtf} which do not intersect with the resolution limit the best resolution of $250\,$lp/mm for the frequency-domain, or $4\, \mu$m for the spatial-domain is assumed, respectively. Hence, for a proper determination of the resolution limit which is directly proportional to the size of the source, another \gls{restarget} with smaller features is required, because with this target it only can be stated that the spot-size has to be, at least in vertical direction, smaller than $4\,\mu$m. In order to quantitatively determine the spot-size with this method, the different \glspl{mtf} usually have to be Fourier transformed to obtain the related \gls{psf} and thus get quantitative values in analogue to the previous section. This transformation was left out due to several reasons. One reason is, the resolution-target only covers a small part of spatial-frequencies. Therefore, the transformation is affected by big errors, which would anyway destroy the validity of the obtained values. Thereby, one source of error is the lack of measurement points at higher spatial frequencies at which their values would drop down zero, which hinders a proper fit of the data and thus a good transformation. Another problem is after the transformation the gap ranging from the origin up to $4\, \mu$m, because the transformed function only covers the range between $4\, \mu$m up to $32\, \mu$m. Hence, an additional fit onto the retrieved part of the PSF has to be done which again induces some error. In addition to this problems, the obtained data of the vertical measurements are influenced by the horizontal extent of the source spot. As mentioned above, the horizontal spot-size lutes the enclosure-lines perpendicularly to the line-pattern into the line-pattern itself, and thus falsifies the contrast values above a power of $25\,$W. This can also be confirmed by sighting of the stitched images in Figure \ref{restargetcombi}. The comparison among themselves yields three main results. First of all, the resolution in vertical direction is again always better than in the corresponding horizontal direction, which means that the spot-size in vertical directions is smaller than in horizontal direction, as discovered with the previous technique. The second, the resolution reduces drastically from the measurement at $30\,$W compared to the measurement at $25\,$W, which also confirms the discovered behaviour in the previous section. Last but not least, the large increase of the horizontal spot-size unfortunately leads to a blurring of the lines perpendicular to the resolution-pattern into the line-pattern, which step by step destroys the accuracy of the measurement above $25\,$W for increasing power (compare Figure \ref{restargetcombi} e)).  

%\clearpage
\section{Summary}\label{sec:ssrsummary}
In this section the results of the knife-edge measurements and the results of the \gls{restarget} measurements are combined to give statements about the source size over the complete covered power range, as each different measurement provided complementary information within a part of the power range. In summary, the measurements and the investigation of the obtained results yielded the possibility to separate the source size with respect to the power in three different parts:
\paragraph{1. Region up to 30\,W:} The behaviour of the spot-size in this region is almost linear with a very small slope, whereupon the size in vertical direction seems to stagnate at a value below $4\, \mu$m up to a power of $20\,$W as one can see in Table \ref{table:reslimit}. As mentioned, the increase at $30\,$W is induced by the much faster increase of the horizontal spot-size. Nevertheless, the horizontal spot-size also stays at almost the same value of $\approx 5.3\, \mu$m and increases abruptly at $30\,$W. Of course these values are rather of qualitative than of quantitative nature, but in correspondence with 'X-RAY WorkX' theses results are also in good agreement with the predictions of the source's data-sheet \citep{DatasheetX} with a resolution of $2\, \mu$m.
\clearpage  
\paragraph{2. Region between 30 and 50\,W:} In this region the size of the source undergoes a steep increase, which is induced by changes of particular source components. As mentioned, the reason is the prevention of melting down the target, due to the increasing intensity of the electrons impacting on the target and the corresponding increase of heat deposition within the impact region. In this small range of $20\,$W the spot-size increases in the vertical direction around a factor of about $2.5$ and in the horizontal direction around a factor of about $4$, respectively, whereupon these factors hold for each particular measurement regardless of particular energy of the measurement cf. Figure \ref{60kvpcomp}.     
\paragraph{3. region above 50\,W:} The third region finally, has again a more moderate increase of the spot-size in each direction. After the rapid change of the focussing of the electron-optic inside the source in region 2, the focussing is again stabilized and is able to slowly increase the spot-size in each direction, because due to the strong increase of the impact area in the first place, the damaging of the target is not any more an issue over a large power range.\\  

In general, the shape of the source spot is in good approximation an ellipse cf Figure \ref{ell60kvpselec}, whereupon the semi-major axis lies in horizontal direction, as the spot-size in horizontal direction is always much bigger than vertical direction caused by the beam geometry cf. Figure \ref{sourceellipse}. Thus the resolution will be much better in vertical direction than in horizontal direction. For a more precise investigation and confirmation of these values a \gls{restarget} with smaller line-patterns, leading a drop down to 0\% of the \gls{mtf}, or a detector with smaller pixel size is required to be able to resolve the edge over the whole power range and hence be able to properly determine the \gls{fwhmg} of the source size. 
